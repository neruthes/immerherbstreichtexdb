\documentclass[11pt,a4paper]{article}
\usepackage[textwidth=39em,tmargin=61mm,bmargin=35mm]{geometry}

\usepackage{calc,amsmath,fontspec,xunicode,tocloft,tabu,paralist,enumitem,eso-pic,graphicx,datetime2,multicol,pgfornament}
\setdefaultleftmargin{2em}{2em}{1em}{1em}{1em}{1em}

\usepackage{datetime2}
\usepackage[hidelinks]{hyperref}
\hypersetup{
	colorlinks=false,
	pdfpagemode=FullScreen
}


\usepackage{xeCJK,xeCJKfntef}
\newcommand{\myvphantom}[0]{\vphantom{QWERTYUIOPASDFGHJKLZXCVBNMqwertyuiopasdfghjklzxcvbnm1234567890ςρθδφγηξλζχψβμ\"A}}
\xeCJKsetup{PunctStyle=plain,RubberPunctSkip=false,CJKglue=\myvphantom\hskip 0pt,CJKecglue=\myvphantom\hskip 0.22em plus 200pt}
\XeTeXlinebreaklocale "zh"
\XeTeXlinebreakskip = 0pt


\setmainfont[Numbers=Lining]{Brygada 1918}
\setromanfont[Numbers=Lining]{Brygada 1918}
\setsansfont[Numbers=Lining]{Nimbus Sans}
\setmonofont{Elite Math}
\setCJKmainfont{Noto Serif CJK SC}
\setCJKromanfont{Noto Serif CJK SC}
\setCJKsansfont{Noto Sans CJK SC}
\setCJKmonofont{Noto Sans CJK SC}

\setlength{\parindent}{2em}
\setlength{\parskip}{8pt}
\linespread{1.22}
\frenchspacing
\pagestyle{empty}

\newcommand{\letterheadgovbody}{Letterhead Gov Body}

\AddToShipoutPictureBG{
	\put(0mm,265mm){%
		\begin{minipage}{\paperwidth}%
			\noindent\begin{center}%
				\noindent\parbox{39em}{
					\rule{\linewidth}{3pt}
					\par\vskip 10pt

					\rmfamily
					\large
					\bfseries
					\scshape
					Regierung des Immerherbstreichs
					\par\vskip 4pt

					\large\mdseries\upshape
					\underline{\myvphantom\letterheadgovbody}\par
				}
			\end{center}
		\end{minipage}%
	}%
	\put(0mm,24mm){%
		\begin{minipage}{\paperwidth}%
			\noindent\begin{center}%
				\rule{39em}{0.5pt}
			\end{center}
		\end{minipage}%
	}%
}


\newcommand{\letterisfrom}[1]{
	\noindent\begin{minipage}{\linewidth}
		\flushright
		\small
		#1\par
	\end{minipage}\par\vskip 20pt
}
\newcommand{\sendletterto}[1]{
	\noindent\begin{minipage}{\linewidth}
		\small
		#1\par
	\end{minipage}\par\vskip 20pt
}
\newcommand{\lettersig}[1]{
	\vskip 30pt
    \vfill
	\noindent
    \hfill
    \begin{minipage}{15em}
		#1\par
	\end{minipage}\par
    \vskip 30pt
}

\usepackage{lipsum}

\newlength{\tmpdima}

\renewcommand{\letterheadgoventity}{Nummer 8}

\newcommand{\examplebox}[2]{
    % argv: caption, content
    \begin{center}%
        \fboxsep=1.5em
        \fbox{%
            \begin{minipage}{\textwidth-4em-3em}%
                #2%
            \end{minipage}%
        }\par
        \footnotesize\sffamily#1
    \end{center}\par
}

\begin{document}
\letterisfrom{
	Nummer 8\\
	8 Nachtrabe Allee\\
	Kramuchdort 10000\\
	Immerherbstreich
}

\sendletterto{
	Broadcast for:\\
	\textbf{All Government Entities of the Reich}\par
	Subject:\\
	\textbf{Government Visual Identity Guidelines Version 1}
}

\noindent
Dear Colleagues,

As the Reichskanzler of the Reich, I am writing this broadcast for all colleagues
in all government entities of the Reich, to announce the latest
Government Visual Identity Guidelines.

These guidelines are applicable for all government entities,
but not for Herbstnachtschloss --- the sovereign may instate its own standards.
Also, Reichsrat, Reichstag, and the judicial system are beyond my control.
They might perhaps follow these guidelines voluntarily, but it is their choice.


\section{Typography}

In most cases, Lora shall be used in all formal government communications.
These communications include letters, announcements, publications, posters, banners, and websites.

For text of smaller sizes (e.g. captions, footnotes, and auxiliary text), it is fine to use Nimbus Sans.
Nimbus Sans may also be used for grand titles and subtitles in posters and similar materials.

The rules above are not applicable for other usages of text, e.g. road signs, information boards, and travel documents.



\section{Heading of Materials}

\subsection{General Pattern}

The top-left corner shall be used to indicate the owner/publisher of the material.
The indication consists of two lines.
The first line shall be bold for static text `\MakeUppercase{Regierung} des \MakeUppercase{Immerherbstreichs}',
where the second word is lowercase and others are uppercase.
The second line shall be underlined for the canonical name of the government entity.
For example, mine is `Nummer 8'.

\examplebox{Example: Generic heading}{
    \small
	\bfseries\MakeUppercase{Regierung} des \MakeUppercase{Immerherbstreichs}\par
	\mdseries\underline{\myvphantom{}Verabreichung für Lebensmitteln und Drogen}
}

\subsection{Lingual Flexibility}

For materials primarily targeting the audience of other languages, e.g. English,
the name of the government entity may be spelled in the target language.
This is particularly appropriate for our diplomatic missions in foreign countries.

\examplebox{Example: Translating the second line}{
    \small
	\bfseries\MakeUppercase{Regierung} des \MakeUppercase{Immerherbstreichs}\par
	\mdseries\underline{\myvphantom{}Embassy of Immerherbstreich at Neoparia Demokratia}
}

In rare cases of extreme specialty, the first line may be translated.

\examplebox{Example: Translating both lines (English)}{
    \small
	\bfseries\MakeUppercase{Government} of \MakeUppercase{Immerherbstreich}\par
	\mdseries\underline{\myvphantom{}Number 8}
}

\examplebox{Example: Translating both lines (Chinese)}{
    \small
	\bfseries 恒秋净土政府\par
	\mdseries\underline{\myvphantom{}首相官邸}
}

\section{Collaboration with Other Organizations}

\subsection{Government as an Equal Partner}

Similar to the generic heading, but it is acceptable to abbreviate the first line into `RIHR'.
Also, it is no longer necessary to put the logo at the top-left corner.
There is greater freedom of placement.

Additionally, when the name of the entity is too long, an acronym may be used.
For example, `Verabreichung für Lebensmitteln und Drogen'
may be written as `VfLD'.

When acronymizing the government entity name,
these words should be omitted:
der, die, das, des, und.
And these words should be preserved as lowercase:
für, von.
When different government entities are having a collision of acronyms,
Nummer 8 shall coordinate a solution for killing ambiguity.

\examplebox{Example: Government as an equal partner}{
    \small
	\settowidth{\tmpdima}{\rmfamily\underline{VfLD}}
	\hfill
	\textsf{SomeBrand1}
	\hfill
	\parbox{\tmpdima}{\textbf{RIHR}\\\underline{\myvphantom VfLD}}
	\hfill
	\myvphantom\par
	\vskip 20pt
	\begin{center}
		\fontspec{QTHelvet-Black}
        \sffamily\bfseries
        \LARGE Some Poster Content
	\end{center}
	\par
}

\subsection{Government as a Contributor}

Essentially equivalent to the previous situation.

\examplebox{Example: Government as a contributor}{
    \small
	\begin{center}
		\fontspec{QTHelvet-Black}
        \sffamily\bfseries
        \LARGE Some Poster Content
	\end{center}
	\par\vskip 20pt
	\settowidth{\tmpdima}{\rmfamily\underline{Nummer 8}}
	\textsf{SomeBrand1}
	\hfill
	\parbox{\tmpdima}{\textbf{RIHR}\\\underline{\myvphantom Nummer 8}}
	\hfill
	\textrm{SomeBrand2}
}






\section*{Afterwords}

With this letter, I have attached a sample poster for reference.

\lettersig{
	Sincerely,\\
	Kjostolv von Zweistein\\
	Reichskanzler\\
	Tue 2023 Feb 07
}
\clearpage
\parindent=0pt





\begin{minipage}{\linewidth}
    \Huge\sffamily\bfseries
    Example Poster Title\\
    With a Second Line\par
\end{minipage}
\vskip 30pt


{\sffamily\LARGE Some Subtitle under the Grand Title}

\vfill
\columnsep=3em
\begin{multicols}{2}
    \lipsum[1-5][1-6]\par
    \lipsum[1-4][1-3]\par
\end{multicols}
\vskip 20pt

{\sffamily\small Some caption at the bottom of the page. Try putting some footnotes here.}






\end{document}
